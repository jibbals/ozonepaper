%\section{Introduction}

Tropospheric ozone constitutes only 10\% of the total ozone column but is an important oxidant and greenhouse gas which is toxic to life, harming natural ecosystems and reducing agricultural productivity.
Over the industrial period, increasing tropospheric ozone has been estimated to exert a radiative forcing (RF) equivalent to a quarter of the CO$_2$ forcing \citep{IPCC_Chapter2}.
Increased ozone in the troposphere is thought to have exerted a warming effect on surface climate \citep{Stevenson2013}.
\citet{Stevenson2013} use an ensemble of 17 models in the Atmospheric Chemistry and Climate Model Intercomparison Project (ACCMIP) to show the tropospheric ozone RF increased by around 365~mWm$^{-2}$ over the industrial period.
Several of these models show a decrease in the tropospheric ozone in the high southern latitudes, due to stratospheric ozone depletion propogated through vertical mixing.
% Moved to thesis, paper focus is more on UT ozone.
%Further tropospheric ozone enhancements are projected to drive reductions in global crop yields equivalent to losses of up to \$USD$_{2000}$ 35 billion per year by 2030 \citep{Avnery2011}, along with detrimental health outcomes equivalent to $\sim$\$USD$_{2000}$11.8 billion per year by 2050 \citep{Selin2009}.
Tropospheric ozone is produced photochemically from NO$_x$ and volatile organic compounds, which have both anthropogenic (fossil fuel, biomass combustion) and natural (wildfires, lightning, biogenic) sources.
In the upper troposphere, downward transport from the ozone-rich stratosphere provides an additional natural source of tropospheric ozone (\citet{Jacobson2000} and references therein).
Understanding and accurately portraying ozone concentrations in the troposphere is important to allow accurate predictions of future climate.
This will become even more important as projections of future climate changes suggest altered the vertical mixing rates, ultra violet index (UVI) and ozone RF \citep{Hegglin2009}.

Stratosphere-to-troposphere transport (STT) primarily impacts the ozone budget in the upper troposphere but can also increase regional surface ozone levels above the legal thresholds set by air quality standards \citep{Danielson1968, Lefohn2011, Langford2012, Zhang2014}.
In the western US, for example, STT events have been shown to contribute up to 30\% of surface ozone in spring \citep{Lin2012}.
Estimates of the overall contribution of STT to tropospheric ozone vary widely (e.g., \citep{Galani2003},  \citet{Stohl2003}, \citet{Stevenson2006}, \citet{Lefohn2011}).
A review of two photochemical models by \citet{Stohl2003} concluded that 25-50\% of the tropospheric ozone column can be attributed to STT events globally, with most contribution in the upper troposphere.
In contrast, \citet{Stevenson2006} found STT was responsible for only $\sim$ 10\% of the tropospheric ozone column (equivalent to $550\pm170$ Tg yr$^{-1}$) in the Atmospheric Chemistry and Climate Model Inter-comparison Project (ACCMIP) simulations, with the remainder produced photochemically.
This wide range in model estimates exists in part because STT is challenging to accurately represent in global models, which are not resolved enough to simulate small scale turbulence.
Observation-based process studies are therefore key in determining the relative frequency of STT events, with models then able to use this to quantify STT impact over large regions.
Ozonesondes are particularly valuable for this purpose as they provide multi-year datasets with high vertical resolution.

%Compared to the tropics, ozone in the extra-tropics has a longer photochemical lifetime(TODO: how long?).
Lower stratospheric and tropospheric ozone concentrations are highly correlated \citep{Terao2008}, suggesting mixing across the tropopause mainly caused by the jet streams over the ocean.
Extratropical STT events most commonly occur during synoptic-scale tropopause folds \citep{Sprenger2003, Tang2012, Frey2015} and are characterised by tongues of high potential vorticity (PV) air descending to low altitudes.
As these tongues become elongated, filaments disperse away from the tongue and mix irreversibly into the troposphere.
The strength (ozone enhancement above background levels), horizontal scale, vertical depth, and longevity of these intruding ozone tongues vary with weather, topography, and season.
STT events have been observed in tropopause folds around both the polar front jet \citep{Vaughan1994, Beekmann1997} and the subtropical jet \citep{Baray2000}.
They are also observed near cut-off lows \citep{Price1993, Wirth1995}, implying the associated turbulent weather may induce stratospheric mixing.

STT events are driven by deep overshooting convection \citep{Frey2015}, tropical cyclones \citep{Das2016} and mid-latitude synoptic scale disturbances (e.g. \citet{Stohl2003, Mihalikova2012}) and are strongly dependent on both season and location. 
While the frequency, seasonality, and impacts of STT events have been well characterised in the tropics and Northern Hemisphere (NH), observational estimates from the Southern Hemisphere (SH) extra-tropics are noticeably absent from the literature. 
Since 1998 NASA has tried to standardise ozonesonde release procedures and improve measurement frequency in the SH through the Southern Hemisphere ADditional OZonesonde (SHADOZ) program (\url{http://croc.gsfc.nasa.gov/shadoz/}).
The SHADOZ ozonesondes were recently used to show increasing upper tropospheric ozone near southern Africa, most likely due to stratospheric mixing \citep{Liu2015, Thompson2014}.
In 2002 \citet{Brinksma2002} examine 5 years of ozonesondes released at Lauder, New Zealand, and show tropospheric ozone depletion following the break up of the polar vortex and dispersion of the ozone hole.
Nonetheless, the role of STT in the SH remains highly uncertain due to the much more limited ozonesonde release sites compared to the NH and temporal sparsity in these datasets  \citep{Liu2015, Thompson2014, Mze2010}. 
%Many of the ozonesonde releases only occur every two to four weeks, ozone intrusion events often last for just a matter of hours \citep{Tang2012}.

%This paragraph doesn't really belong, I moved some of it's text.
% Ozonesondes are useful for looking at specific locations with high resolution, and in this work they provide an estimate of both STT occurrence rates and STT ozone flux.
%At these discrete locations, this information can be used in conjunction with regional-scale information in order to estimate large-scale impacts of STT on tropospheric ozone. Here, the 
%GEOS-Chem chemical transport model (CTM) is used to provide the regional-scale ozone concentrations.

% Tropospheric ozone is lost via chemical destruction and dry deposition, estimated to be $4700\pm700$ Tg yr$^{-1}$ and $1000\pm200$ Tg yr$^{-1}$, respectively \citep{Stevenson2006}. 

%Hegglin, M. I., and T. G. Shepherd (2009), Large climate-induced changes in ultraviolet index and stratosphere-to-troposphere ozone flux, Nature Geosci, 2(10), 687–\selectlanguage{english}691, doi:10.1038/NGEO604.

% AIMs paragraph
Here, we characterise the seasonal cycle of STT events and quantify their contribution to the tropospheric ozone budget using nearly a decade of ozonesonde observations from three locations around the Southern Ocean spanning latitudes from 38$^{\circ}$S - 69$^{\circ}$S. 
In Section 2 we describe the observations and the methods used to identify STT events and to relate STT occurrence to meteorological events.
Section 3 provides our newly derived climatologies of STT frequency, seasonality, intrusion altitude, and depth, and Section 4 uses these new climatologies to evaluate upper tropospheric ozone in a global chemical transport model (GEOS-Chem). 
Finally, in Section 5 we use the observations and the model to estimate the overall contribution of STT events to total tropospheric ozone in the high southern latitudes.
