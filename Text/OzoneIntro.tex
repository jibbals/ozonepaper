%\section{Introduction}

Tropospheric ozone constitutes only 10\% of the total ozone column but is an important oxidant and greenhouse gas which is toxic to life, harming natural ecosystems and reducing agricultural productivity.
Over the industrial period, increasing tropospheric ozone has been estimated to exert a radiative forcing (RF) of 365~mWm$^{-2}$  \citep{Stevenson2013}, equivalent to a quarter of the CO$_2$ forcing \citep{IPCC_Chapter2}. 
While much tropospheric ozone is produced photochemically from anthropogenic and natural precursors, %from NO$_x$ and volatile organic compounds, which have both anthropogenic (fossil fuel, biomass combustion) and natural (wildfires, lightning, biogenic) sources.
downward transport from the ozone-rich stratosphere provides an additional natural source of ozone that is particularly important in the upper troposphere (\citet{Jacobson2000} and references therein).
The contribution of this source to overall tropospheric ozone budgets remains uncertain, especially in the southern hemisphere.
While models show decreasing tropospheric ozone due to stratospheric ozone depletion propagated to the upper troposphere through vertical mixing \citep{Stevenson2013}, recent work based on the Southern Hemisphere ADditional OZonesonde (SHADOZ) network suggests increasing upper tropospheric ozone near southern Africa, most likely due to stratospheric mixing \citep{Liu2015, Thompson2014}.
Uncertainties in the various processes which produce tropospheric ozone limit predictions of future ozone-induced radiative forcing.
Here we use a multi-year record of ozonesonde observations from sites in the southern hemisphere extratropics, combined with a global model, to better characterise the impact of stratospheric ozone on the tropospheric ozone budget in the southern hemisphere.
%Understanding and accurately portraying ozone concentrations in the troposphere is important to allow accurate predictions of future climate.
%This will become even more important as projections of future climate changes suggest altered vertical mixing rates, ultra violet index (UVI) and ozone RF \citep{Hegglin2009}.
% Doesn't really belong in first intro paragraph - maybe in conclusions? -jaf

Stratosphere-to-troposphere transport (STT) primarily impacts the ozone budget in the upper troposphere but can also increase regional surface ozone levels above the legal thresholds set by air quality standards \citep{Danielson1968, Lelieveld2009, Lefohn2011, Langford2012, Zhang2014, Lin2015}.
In the western US, for example, deep STT events during spring can add 20-40~ppbv of ozone to the ground-level ozone concentration, which can provide over half the ozone needed to exceed the standard set by the U.S. Environmental Protection Agency \citep{Lin2012, Lin2015}.
Another important region of STT is the Middle East, where surface ozone excedes values of 80~ppbv in summer, with 10~ppbv from STT contribution \citep{Lelieveld2009}.
Estimates of the overall contribution of STT to tropospheric ozone vary widely \citep[e.g.,][]{Galani2003, Stohl2003, Stevenson2006, Lefohn2011}.
Early work based on two photochemical models showed that 25-50\% of the tropospheric ozone column can be attributed to STT events globally, with most contribution in the upper troposphere \citep{Stohl2003}.
In contrast, a more recent analysis of the Atmospheric Chemistry and Climate Model Inter-comparison Project (ACCMIP) simulations by \citet{Young2013} found STT is responsible for $540\pm140$~Tg yr$^{-1}$, equivalent to $\sim$11\% of the tropospheric ozone column, with the remainder produced photochemically \citep{Monks2015}.
This wide range in model estimates exists in part because STT is challenging to accurately represent, and better model resolution is necessary to simulate small scale turbulence.
Observation-based process studies are therefore key in determining the relative frequency of STT events, with models then able to quantify STT impact over large regions.
Ozonesondes are particularly valuable for this purpose as they provide multi-year datasets with high vertical resolution.

Lower stratospheric and upper tropospheric ozone concentrations are highly correlated \citep{Terao2008}, suggesting mixing across the tropopause mainly caused by the jet streams over the ocean.
Extratropical STT events most commonly occur during synoptic-scale tropopause folds \citep{Sprenger2003, Tang2012, Frey2015} and are characterised by tongues of high potential vorticity (PV) air descending to low altitudes.
As these tongues become elongated, filaments disperse away from the tongue and mix irreversibly into the troposphere.
STT can also be induced by deep overshooting convection \citep{Frey2015}, tropical cyclones \citep{Das2016} and mid-latitude synoptic scale disturbances (e.g., \citet{Stohl2003, Mihalikova2012}). 
STT events have been observed in tropopause folds around both the polar front jet \citep{Vaughan1994, Beekmann1997} and the subtropical jet \citep{Baray2000}.
A big influence on the high surface ozone concentrations over the eastern Mediterranean is stratospheric mixing and anticyclonic subsidence \citep{Zanis2014}. 
They are also observed near cut-off lows \citep{Price1993, Wirth1995}, so both regional weather patterns and stratospheric mixing are important to understand for STT analysis.

%STT events are driven by deep overshooting convection \citep{Frey2015}, tropical cyclones \citep{Das2016} and mid-latitude synoptic scale disturbances (e.g., \citet{Stohl2003, Mihalikova2012}) and are strongly dependent on both season and location. 
The strength (ozone enhancement above background levels), horizontal scale, vertical depth, and longevity of these intruding ozone tongues vary with weather, topography, and season.
%Because of their dependence on meteorological phenomena, STT events are strongly dependent on both season and location.
While the frequency, seasonality, and impacts of STT events have been well characterised in the tropics and northern hemisphere (NH), observational estimates from the southern hemisphere (SH) extra-tropics are noticeably absent in the literature. 
%Since 1998 NASA has tried to standardise ozonesonde release procedures and improve measurement frequency in the SH through the Southern Hemisphere ADditional OZonesonde (SHADOZ) program (\url{http://croc.gsfc.nasa.gov/shadoz/}).
% moved reference earlier
%Recent work based on the  Southern Hemisphere ADditional OZonesonde (SHADOZ) ozonesondes suggests increasing upper tropospheric ozone near southern Africa, most likely due to stratospheric mixing \citep{Liu2015, Thompson2014}.
% jaf
%In 2002 \citet{Brinksma2002} examine 5 years of ozonesondes released at Lauder, New Zealand, and show tropospheric ozone depletion following the break up of the polar vortex and dispersion of the ozone hole.
The role of STT in the SH remains highly uncertain due to the much more limited data availability compared to the NH and the temporal sparsity of these datasets \citep{Mze2010, Thompson2014, Liu2015}. 
%Many of the ozonesonde releases only occur every two to four weeks, ozone intrusion events often last for just a matter of hours \citep{Tang2012}.

% AIMs paragraph
Here, we characterise the seasonal cycle of STT events and quantify their contribution to the SH extratropical tropospheric ozone budget using nearly a decade of ozonesonde observations from three locations around the Southern Ocean spanning latitudes from 38$^{\circ}$S-69$^{\circ}$S. 
In Section 2 we describe the observations and the methods used to identify STT events and to relate STT occurrence to meteorological events.
Section 3 provides our newly derived climatologies of STT frequency, seasonality, intrusion altitude, and depth, and Section 4 uses these new climatologies to evaluate upper tropospheric ozone in a global chemical transport model (GEOS-Chem). 
Finally, in Section 5 we use the observations and the model to estimate the overall contribution of STT events to total tropospheric ozone in the high southern latitudes.
