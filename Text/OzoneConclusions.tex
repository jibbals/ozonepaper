  
%\section{Conclusions}
  
Stratosphere-to-troposphere transport can be a major source of ozone to the remote free troposphere, but the occurrence and influence of these STT events remains poorly quantified in the southern hemisphere (SH) extratropics.
Ozonesonde observations in the Southern Hemisphere provide a satellite-independent quantification of the frequency of STT events, as well as an estimate of their impact and source.
%The frequency and amount of ozone descending from the stratosphere into the troposphere can be estimated from the long time series of tropospheric ozone profiles.
Using almost ten years of ozonesonde profiles over the southern high latitudes, we have quantified the frequency, seasonality, and altitude distributions of STT events in the SH mid-latitudes. 
By combining this information with ozone column estimates with a global chemistry transport model, we provide a first, conservative estimate of the influence of STT events on tropospheric ozone over the Southern Ocean.

Our method involves applying a Fourier bandpass filter to the measured ozone profiles to determine transport events.
The filter removes seasonal influences and allows clear detection of ozone-enhanced tongues of air in the troposphere.
By setting empirically-derived threshholds, ozonesondes can clearly distinguish tropospheric ozone enhancements that are separated from the stratosphere.
This method is sensitive to some parameters involved in the calculation, however for our sites we see no false positive detections of transport events.

Examination of STT events at three sites spanning the SH extra-tropics (38$^{\circ}$S, 55$^{\circ}$S, and 69$^{\circ}$S) shows a distinct seasonal cycle.
All three sites show a summer (DJF) maximum and winter (JJA) minimum, although the seasonal amplitude is less apparent at the Antarctic site (Davis), as events (likely due to polar jet stream-caused turbulence) are also detected regularly in winter and spring.  %The cause of these ozone enhancements is examined through the use of satellite CO data (to determine fire influence) and ERA-I meteorological data.
Comparison with ERA-Interim reanalysis data suggests the majority of events are caused by turbulent weather in the upper troposphere due to low pressure fronts, followed by cut-off low pressure systems.

A comparison of ozonesonde-measured ozone profiles against those simulated by the GEOS-Chem global chemical transport model is performed.
Vertical integration of the ozonesonde-observed ozone provides the tropospheric ozone column.
It also allows an estimate of the flux associated with each event, although this is a conservative lower limit.
The seasonal features in the ozonesonde record of tropospheric ozone columns are well represented by the model.
Simulated profiles generally can not identify STT events due to their lower vertical resolution.
Simulated tropospheric column ozone is combined with ozonesonde data to provide a first estimate of the total contribution of STT events to tropospheric ozone over the southern ocean.
Across all months, we conservatively estimate that the ozone enhancement due to STT events at our three sites ranges from $1$ to $6 \times 10^{16}$ molecules cm$^{-2}$.
This method implies that STT events can only explain $\sim$3\% of the global net downward flux of ozone, a value that is likely too low.
Replacing the sonde-derived estimate of the ozone fraction associated with each event with a value derived from a global model \citep{Terao2008} increases this estimate by an order of magnitude.

Estimating STT flux using ozonesonde data alone remains challenging; however, the very high vertical resolution provided by ozonesondes means they are capable of detecting STT events that models, reanalyses, and satellites may not. 
Further work is needed to more accurately translate these ozonesonde measurements into STT ozone fluxes, particularly in the SH where data are sparse and STT is likely to be a major contributor to upper tropsopheric ozone in some regions.
More frequent ozonesonde releases at SH sites would facilitate development of better STT flux estimates for this region.
%As the climate alters, an understanding of mixing between the stratosphere and troposphere over the southern ocean will help diagnose potential hazards.
%Seasonality and impact of stratospheric ozone are changing over time, furter analysis of this long term set of ozonesonde data will be useful into the future.

