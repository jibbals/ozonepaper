  
%\section{Conclusions}
  
Stratosphere-to-troposphere transport (STT) can be a major source of ozone to the remote free troposphere, but the influence of these STT events remains poorly quantified in the southern hemisphere (SH) extratropics.
Ozonesonde observations in the Southern Hemisphere provide a potential satellite-independent quantification of the ozone fluxes associated with STT events.
%The frequency and amount of ozone descending from the stratosphere into the troposphere can be estimated from the long time series of tropospheric ozone profiles.
Using almost ten years of ozonesonde profiles over the southern high latitudes, we have quantified the frequency, seasonality, and altitude distributions of STT events in the SH mid-latitudes. 
By combining this information with ozone column estimates with a global chemistry transport model, we provide a first, conservative estimate of the influence of STT events on tropospheric ozone over the Southern Ocean.

%Using almost ten years of ozonesonde profiles over the southern high latitudes, a clear summer peak is seen for STT occurrences at our three sites (38$^{\circ}$S, 55$^{\circ}$S, and 69$^{\circ}$S).

Our method involves applying a Fourier bandpass filter to the measured ozone profiles to identify STT events.
The filter removes seasonal influences and allows clear detection of ozone-enhanced tongues of air in the troposphere.
By setting empirical checks, ozonesonde vertical profiles clearly show tropospheric ozone enhancements that are separated from the stratosphere.
This method is sensitive to some parameters involved in the calculation, and likely provides a lower limit on the frequency of STT events.%but can determines clear tropospheric ozone peaks from our record of ozonesondes.

Our new climatology of STT events at three sites spanning the SH extra-tropics (38$^{\circ}$S, 55$^{\circ}$S, and 69$^{\circ}$S shows a distinct seasonal cycle.
All three sites show a summer (DJF) maximum and winter (JJA) minimum, although the seasonal amplitude is less apparent at the Antarctic site (Davis), as events are also detected regularly in winter and spring.  %The cause of these ozone enhancements is examined through the use of satellite CO data (to determine fire influence) and ERA-I meteorological data.
Comparison with ERA-Interim reanalysis data suggests the majority of events are caused by turbulent weather in the upper troposphere due to low pressure fronts, followed by cut-off low pressure systems.


Vertical integration of the observed ozone enhancement also allows a rough estimate of the flux associated with each event, although this is a conservative lower limit.
We find that on average STT events are responsible for 3\% of the total tropospheric ozone column, although this varies seasonally.
Across all months, we conservatively estimate that the ozone enhancement due to STT events at our three sites ranges from $1$ to $6 \times 10^{16}$ molecules cm$^{-2}$.
\citet{Hegglin2009} estimate that climate change will lead to increased STT of the order of 30 (121) Tg yr$^{-1}$ relative to 1965 in the Southern (Northern) Hemisphere due to an acceleration in the Brewer Dobson circulation.
Seasonality and impact of stratospheric ozone may be changing over time, furter analysis of ozonesondes may be useful into the future.

%GEOS-Chem estimates of the tropospheric ozone column match fairly well with the ozonesonde observations, with least squares regression correlation coefficient (r$^2$) of 0.38, 0.18, 0.37, for Davis, Macquarie, and Melbourne respectively.
%Using reduced major axis regression (which does not assume that there is no error in the measurements) there is a near one to one slope at all three sites, with or without removing the seasonality of both GEOS-Chem and the ozonesondes.
%Our STT detection algorithm is unsuitable for GEOS-Chem output, as the model averages over 2$^{\circ}$ latitude by 2.5$^{\circ}$ longitude grid boxes, with a lower vertical resolution (especially in the upper troposphere).

We find that the major features of the observed time series of tropospheric ozone columns can be represented by the GEOS-Chem global chemical transport model, although the vertical resolution is not sufficient for the model to simulate STT events. 
Instead, we combine simulated total column ozone from the model with STT flux amounts and frequencies derived from the measurements to provide a first estimate of the total contribution of STT events to tropospheric ozone over the southern ocean.
This method implies that STT events can only explain $\sim$3\% of the global net downward flux of ozone, a value that is likely much too low. 
Replacing the sonde-derived estimate of the ozone fraction associated with each event with a value derived from a global model \citep{Terao2008} increases this estimate by an order of magnitude.

Estimating STT flux using ozonesonde data alone remains challenging; however, the very high vertical resolution provided by ozonesondes means they are capable of detecting STT events that models, reanalyses, and satellites may not. 
Further work is needed to more accurately translate these ozonesonde measurements into STT ozone fluxes, particularly in the SH where data are sparse and STT is likely to be a major contributor to upper tropsopheric ozone in some regions.
More frequent ozonesonde releases at SH sitewould facilitate development of better STT flux estimates for this region.


%Due to their high vertical resolution ozonesondes are able to detect STT events where reanalysis datasets or models may not.

%    Further work needs to be done to quantify STT flux using solely ozonesonde data, more frequent ozonesonde releases would make this easier.
