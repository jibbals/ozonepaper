  We develop a quantitative method to identify Stratosphere to Troposphere Transport events (STTs) from ozonesonde profiles. 
  Using this method we estimate the seasonality and quantity of ozone transported across the tropopause over Melbourne ($38^\circ$S), Macquarie Island ($54^\circ$S), and Davis ($69^\circ$S).
  STT seasonality is determined by two distinct methods using 7--9 years of ozone profiles from each site.
  The primary method we focus on here shows STT events frequently occurring during summer above all three sites.
  The majority of tropospheric ozone peaks from by STT events occur within 3~km below the tropopause at Melbourne and Macquarie Island, and within 2~km below the tropopause at Davis.
  Overall, the fraction of total tropospheric ozone attributed to STT events is 2 – 4\% at each site, however, during individual events, an STT event can contribute more than 10\% of the total tropospheric ozone at that time.
  The meteorological cause of STT events is postulated through visual inspection of ERA-I weather data, with the majority of events caused by low pressure frontal systems at all three sites.
  Ozone enhancements caused by transported biomass burning plumes are flagged after analysis of CO column measurements from the AIRS satellite.
  We use the GEOS-Chem model to understand our point-source ozonesonde results in a 3-dimensional context.
  The GEOS-Chem model is run with active stratospheric chemistry, and is too coarsely resolved in the vertical dimension to determine STTs.
  Simulated seasonal cycles of tropospheric ozone are well matched at all three sites although vertical profile averages have some bias in the troposphere compared with ozonesondes.
  A conservative estimate of yearly tropospheric ozone flux due to STTs are calculated using the simulated tropospheric ozone column between 35$^\circ$S and 75$^\circ$S of $3.2\times10^{16}$ molecules cm$^{-2}$ yr$^{-1}$.
  Using an assumed STT impact (of 30\%) over the southern ocean rather than our conservative calculation increases the flux estimate to $\sim$30.2~Tg yr$^{-1}$.
  