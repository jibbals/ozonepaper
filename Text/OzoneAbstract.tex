  Stratosphere-to-troposphere transport (STT) provides an important natural source of ozone to the upper troposphere, but the characteristics of STT events in the southern hemisphere extratropics and their contribution to the regional tropospheric ozone budget remain poorly constrained.
  Here, we develop a quantitative method to identify STT events from ozonesonde profiles. 
  Using this method we estimate the seasonality and quantity of ozone transported across the tropopause over Melbourne ($38^\circ$S), Macquarie Island ($54^\circ$S), and Davis ($69^\circ$S).
  STT seasonality is determined by two distinct methods using 7--9 years of ozone profiles from each site.
  Using a bandpass filter provides clear detection STT events, which are seen to be seasonal with a maximum during summer and minimum during winter above all three sites.
  The majority of tropospheric ozone enhancements from STT events occur within 3~km of the tropopause at Melbourne and Macquarie Island, and within 2~km of the tropopause at Davis.
  The mean fraction of total tropospheric ozone attributed to STT during STT events is 2–4\% at each site; however, during individual events over 10\% of the total tropospheric ozone may be directly transported from the stratosphere.
  Coincident ERA-Interim reanalysis data suggest the majority of events at all sites are caused by low pressure frontal systems.
  %Ozone enhancements caused by transported biomass burning plumes are flagged using an analysis of CO column measurements from the AIRS satellite.
  %The effects of African and South American biomass burning are apparent during the austral autumn and winter months.
  Ozone enhancements caused by biomass burning plumes transported from Africa and South America are apparent during austral winter and spring. 
  To provide regional context for the ozonesonde observations, we use the GEOS-Chem chemical transport model, which is too coarsely resolved to distinguish STT events but is able to accurately simulate the seasonal cycle of tropospheric ozone columns over the three southern hemisphere sites.
  %The GEOS-Chem model is run with active stratospheric chemistry, however is too coarsely resolved in the vertical and horizontal dimensions to determine STTs.
  %Simulated seasonal cycles of tropospheric ozone are well matched at all three sites although vertical profile averages have some (non-significant) bias in the troposphere compared with ozonesondes.
  Combining the ozonesonde-derived STT event characteristics with the simulated tropospheric ozone columns from GEOS-Chem, we conservatively estimate that the yearly tropospheric ozone flux over the Southern Ocean due to STT events is $\sim3.2\times10^{16}$ molecules cm$^{-2}$ yr$^{-1}$.
  %A conservative estimate of yearly tropospheric ozone flux due to STTs are calculated using the simulated tropospheric ozone column between 35$^\circ$ S and 75$^\circ$ S of $3.2\times10^{16}$ molecules cm$^{-2}$ yr$^{-1}$.
  This value is significantly lower than expected from previous global estimates due to the conservative nature of several components of our calculation, in particular the contribution of STT to total tropospheric ozone during an event (STT impact).
  Using an assumed STT impact of 35\% based on prior modelling studies rather than our observational estimate of 2--4\% increases the estimated Southern Ocean flux to $\sim$29.5~Tg yr$^{-1}$.
  Despite lingering uncertainties in scaling ozonesonde measurements to regional values, ozonesonde datasets provide a useful tool for STT detection, and the analysis methods described in this paper could be applied to many existing long-term records.
  