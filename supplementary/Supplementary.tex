%% Copernicus Publications Manuscript Preparation Template for LaTeX Submissions
%% ---------------------------------
%% This template should be used for copernicus.cls
%% The class file and some style files are bundled in the Copernicus Latex Package, which can be downloaded from the different journal webpages.
%% For further assistance please contact Copernicus Publications at: production@copernicus.org
%% http://publications.copernicus.org/for_authors/manuscript_preparation.html
%% Please use the following documentclass and journal abbreviations for discussion papers and final revised papers.

\documentclass[acp, manuscript]{copernicus} % single column manuscript: used for both discussion and submission

% shortcut commands for one or two column figures
%\newcommand{\igone}[1]{\includegraphics[width=8.3cm]{#1}}
%\newcommand{\igtwo}[1]{\includegraphics[width=12.0cm]{#1}}

\begin{document}
\title{Supplementary}

% \Author[affil]{given_name}{surname}
\Author[1]{Jesse W.}{Greenslade}
\Author[2,3]{Simon P.}{Alexander}
\Author[4,5]{Robyn}{Schofield}
\Author[1,6]{Jenny A.}{Fisher}
\Author[2,3]{Andrew K.}{Klekociuk}

\affil[1]{Centre for Atmospheric Chemistry, School of Chemistry, University of Wollongong}
\affil[2]{Australian Antarctic Division, Hobart}
\affil[3]{Antarctic Climate and Ecosystems Co-operative Research Centre, Hobart, Australia}
\affil[4]{School of Earth Sciences, University of Melbourne}
\affil[5]{ARC Centre of Excellence for Climate System Science, University of New South Wales}
\affil[6]{School of Earth \& Environmental Sciences, University of Wollongong}

\runningtitle{Southern Hemisphere stratospheric ozone intrusions}
\runningauthor{Greenslade et al.}
\correspondence{Jesse Greenslade (jwg366@uowmail.edu.au)}

\received{}
\pubdiscuss{} %% only important for two-stage journals
\revised{}
\accepted{}
\published{}

%% These dates will be inserted by Copernicus Publications during the typesetting process.
\firstpage{1}
\maketitle


%\begin{abstract}
%\end{abstract}

\section{Southern Ocean extrapolation}

\subsection{Outline}
  We use simulated tropospheric ozone columns from GEOS-Chem to extrapolate the ozonesonde-based estimates to the entire Southern Ocean region, defined here as 35$^{\circ}$ S-75$^{\circ}$ S to encompass our three measurement sites. 
  To do so, we multiply the monthly likelihoods of STTs (fraction of ozonesonde releases for which an STT event was detected, per month, \textit{f}) by the monthly mean fraction of the ozone column attributed to STT, \textit{I}) and by the monthly mean tropospheric ozone column over the Southern Ocean (from the GEOS-Chem multi-year mean, $\Omega_{SO_{O_3}}$), expressed simply as Flux$= \Omega_{SO_{O_3}} \times f \times I$.
  The monthly values of each term in this equation are shown in Fig. \ref{fig:SOExtrapolation} (upper panel).
  A more spatially resolved estimate could be determined by dividing the Southern Ocean region into longitudinal and latitudinal bins for calculating the average $\Omega_{SO_{O_3}}$ from GEOS-Chem, applying latitudinal gradients to $f$ and $I$ based on their values at the three sonde release sites, and adding longitudinal variability due to seasonal stratospheric wind jet streams \citep{Baray2012,Skerlak2015}, but this is beyond the scope of this work and in any case would not address the limitations of the estimate provided below.

  To give an idea of sensitivity to zonal definition, a change of the bounding latitudes by 5$^{\circ}$ in either direction at either end changes the average simulated tropospheric ozone by -7 to 9\%.
  It is worth noting that this extrapolation is very simplistic and is performed as a simple example of how the seasonal ozone STT calculations could be used.
  
  
\subsection{Code}

\subsection{Results}


\bibliographystyle{copernicus}
\bibliography{bibliography/Ozone.bib}

\end{document}
